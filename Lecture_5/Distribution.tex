% \documentclass[handout]{beamer} % Suppress the pauses to print as handout
\documentclass[aspectratio=169, 11pt]{beamer}
% \documentclass[11pt]{beamer} % Comment the line above and use this line for 4:3 aspect ratio screens
\usepackage{advdate}
\usepackage{amsmath}
\usepackage{amsthm}
\usepackage{amsfonts}
\usepackage{amssymb}
\usepackage{array}
\usepackage{color}
\usepackage{helvet}
\usepackage{hyperref}
\usepackage{mathpazo}
\usepackage{multirow}

\usetheme{CambridgeUS}
\usecolortheme{dolphin}
\usefonttheme[onlymath]{serif}

% Define colors considering color blindness
% Use MATLAB default color palette
\definecolor{blue}{rgb}{0, 0.447, 0.741}
\definecolor{red}{rgb}{0.85, 0.325, 0.098}
\definecolor{yellow}{rgb}{0.929, 0.694, 0.125}
\definecolor{purple}{rgb}{0.494, 0.184, 0.556}
\definecolor{green}{rgb}{0.466, 0.674, 0.188}
\definecolor{cyan}{rgb}{0.301, 0.745, 0.933}
\definecolor{magenta}{rgb}{0.635, 0.078, 0.184}
\definecolor{amber}{rgb}{1.0, 0.75, 0.0}

% Self-defined background and frametitle colors
\definecolor{MyBackground}{RGB}{255, 255, 230}
% Uncomment the following line for plain white background
% \definecolor{MyBackground}{RGB}{255, 255, 255}
\setbeamercolor{background canvas}{bg=MyBackground!40}
\setbeamercolor{frametitle}{bg=MyBackground!80}
% Customize word colors
\setbeamercolor{frametitle}{fg=blue}
\setbeamercolor{title}{fg=blue}
\setbeamercolor{itemize item}{fg=blue}
\setbeamercolor{itemize subitem}{fg=blue}
\setbeamercolor{enumerate item}{fg=blue}
\setbeamercolor{enumerate subitem}{fg=blue}
\setbeamercolor{button}{bg=MyBackground,fg=blue}

% Page number on the bottom
\setbeamertemplate{footline}[frame number]
\setbeamerfont{footline}{size=\fontsize{10}{12}\selectfont}
\setbeamercolor{page number in head/foot}{fg=blue}
% No navigation symbols
\setbeamertemplate{navigation symbols}{}
% No headlines
\setbeamertemplate{headline}{}
% Appendix slides don't count
\newcommand{\backupbegin}{
   \newcounter{framenumberappendix}
   \setcounter{framenumberappendix}{\value{framenumber}}
}
\newcommand{\backupend}{
   \addtocounter{framenumberappendix}{-\value{framenumber}}
   \addtocounter{framenumber}{\value{framenumberappendix}}
}
% Add section title at the beginning of each part
\AtBeginSection[]{
  \begin{frame}[noframenumbering, plain]
  \vfill
  \centering
    \begin{beamercolorbox}[sep=8pt, center, shadow=true, rounded=true]{title}
      \usebeamerfont{frametitle}\insertsectionhead\par%
    \end{beamercolorbox}
  \vfill
  \end{frame}
}

% New theorem environments
\newtheorem{thm}{Theorem}
\newtheorem{lem}[thm]{Lemma}
\newtheorem{prop}[thm]{Proposition}

\beamersetuncovermixins{\opaqueness<1>{25}}{\opaqueness<2->{15}}

% Set pause options to be invisible instead of transparent
\setbeamercovered{invisible}

\begin{document}
\title{Lecture 5 \\ Approximating Distribution}
\author[Wu]{Peifan Wu}
\date{UBC}

\begin{frame}
\titlepage
\end{frame}

\begin{frame}
\frametitle{Approaches}
  \begin{itemize}
    \item[--] Continuum of households makes the wealth distribution an \textcolor{red}{infinite dimensional object} in the state space
    \item[--] Need approximation
    \bigskip
    \item[--] Several approaches beyond Monte-Carlo simulation
    \begin{itemize}
      \item[1.] Discretization of the density function
      \item[2.] Eigenvector method
      \item[3.] Parametrizing in exponential family
    \end{itemize}
  \end{itemize}
\end{frame}

\begin{frame}
\frametitle{Discretization of the Invariant Density Function}
  \begin{itemize}
    \item[--] The grid should be finer than the one we used to compute the optimal savings rule
    \bigskip
    \item[--] Choose some initial density function $\lambda^{0}\left(a_{k},y_{i}\right)$
    \bigskip
    \item[--] For every $\left(a_{k},y_{j}\right)$ on the grid:
    \begin{align*}
      \lambda^{1}\left(a_{k},y_{j}\right) & =\sum_{y_{i}\in Y}\pi_{ij}\sum_{m\in M_{i}}\frac{a_{k+1}-g\left(a_{m},y_{i}\right)}{a_{k+1}-a_{k}}\lambda^{0}\left(a_{m},y_{i}\right)\\
      \lambda^{1}\left(a_{k+1},y_{j}\right) & =\sum_{y_{i}\in Y}\pi_{ij}\sum_{m\in M_{i}}\frac{g\left(a_{m}.y_{i}\right)-a_{k}}{a_{k+1}-a_{k}}\lambda^{0}\left(a_{m},y_{i}\right)
    \end{align*}
    where $M_{i}=\left\{ m=1,\ldots,N|a_{k}\leqslant g\left(a_{m},y_{i}\right)\leqslant a_{k+1}\right\}$
    \bigskip
  \end{itemize}
\end{frame}

\begin{frame}
\frametitle{Lottery Rule}
  \begin{itemize}
    \item[--] We can think of this way of handling the discrete approximation to the density function as \textcolor{red}{forcing the agents in the economy to play a lottery}.
    \bigskip
    \item[--] If the optimal policy is to save $a'\in\left[a_{k},a_{k+1}\right]$, then with probability $\frac{a_{k+1}-a'}{a_{k+1}-a_{k}}$ you go to $a_{k}$, and with probability $\frac{a'-a_{k}}{a_{k+1}-a_{k}}$ you go to $a_{k+1}$.
  \end{itemize}
\end{frame}

\begin{frame}
\frametitle{Eigenvector Method}
  \begin{itemize}
    \item[--] The invariant pdf for a Markov transition matrix $Q$ is $\lambda^{*}$ that satisfies $\lambda^{*}Q=\lambda^{*}$
    \bigskip
    \item[--] \textcolor{red}{Perron-Frobenius Theorem}: $Q$ has a unique dominant eigenvalue $\epsilon=1$ such that
    \begin{itemize}
      \medskip
      \item[--] its associated eigenvector has all positive entries
      \medskip
      \item[--] all other eigenvalues are smaller than $\epsilon$ in absolute value
      \medskip
      \item[--] $Q$ has no other eigenvector with all non-negative entries
    \end{itemize}
    \bigskip
    \item[--] This eigenvector (renormalized so that it sums to one) is the unique invariant distribution
  \end{itemize}
\end{frame}

\begin{frame}
\frametitle{Parameterized Distribution}
  \begin{itemize}
    \item[--] For smooth and unimodal distributions we can approximate with parameterized functions in the \textcolor{red}{exponential families}
    \bigskip
    \item[--] Central moments are sufficient statistics of the distribution
  \end{itemize}
\end{frame}

\begin{frame}
\frametitle{Parameterized Distribution}
  \begin{itemize}
    \item[--] To approximate a one-dimensional distribution on $k$, assume the density function form
    \[
      P\left(k,\rho\right)=\rho_{0}\exp\left(\begin{array}{c}
      \rho_{1}\left(k-m_{1}\right)\\
      +\rho_{2}\left[\left(k-m_{1}\right)^{2}-m_{2}\right]\\
      +\rho_{3}\left[\left(k-m_{1}\right)^{3}-m_{3}\right]\\
      \cdots\\
      +\rho_{N}\left[\left(k-m_{1}\right)^{N}-m_{N}\right]
      \end{array}\right)
    \]
    \item[--] $\rho$ -- parameters to pin down, $m$ -- central moments
  \end{itemize}
\end{frame}

\begin{frame}
\frametitle{Parameterized Distribution}
  \begin{itemize}
    \item[--] We choose $\rho$ such that moment condition holds
    \begin{align*}
      \int\left(k-m_{1}\right)P\left(k,\rho\right)\mathrm{d}k & =0\\
      \int\left[\left(k-m_{1}\right)^{2}-m_{2}\right]P\left(k,\rho\right)\mathrm{d}k & =0\\
       & \cdots\\
      \int\left[\left(k-m_{1}\right)^{N}-m_{N}\right]P\left(k,\rho\right)\mathrm{d}k & =0
    \end{align*}
    \item[--] These condition happen to be the optimal condition of
    \[
      \min_{\rho_{1},\ldots,\rho_{N}}\int P\left(k,\rho\right)\mathrm{d}k
    \]
    \item[--] And the distribution has to be normalized to unity
    \[
      \frac{1}{\rho_{0}}=\int P\left(k,\rho\right)\mathrm{d}k
    \]
  \end{itemize}
\end{frame}

\begin{frame}
\frametitle{Parameterized Distribution}
  Steps to compute invariant distribution:
  \begin{itemize}
    \item[1.] Choose capital grid $\left{k_{i}\right}$ and guess initial central moments $m_{j,0}$
    \item[2.] Solve the minimization problem for $\rho$ with $m_{j,0}$
    \[
      \frac{1}{\rho_{0}}=\min_{\rho_{1},\ldots,\rho_{N}}\int P\left(k,\rho\right)\mathrm{d}k
    \]
    \item[3.] With current distribution $P\left(k,\rho\right)$, compute moments using policy function $g(k)$
    \begin{align*}
      m_{1,1} & =\int g\left(k\right)P\left(k,\rho\right)\mathrm{d}k\\
      m_{2,1} & =\int\left(g\left(k\right)-m_{1,1}\right)^{2}P\left(k,\rho\right)\mathrm{d}k\\
      & \cdots\\
      m_{N,1} & =\int\left(g\left(k\right)-m_{1,1}\right)^{N}P\left(k,\rho\right)\mathrm{d}k
    \end{align*}
    \item[4.] Check convergence of $m$
  \end{itemize}
\end{frame}

\end{document}
